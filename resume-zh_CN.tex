% !TEX TS-program = xelatex
% !TEX encoding = UTF-8 Unicode
% !Mode:: "TeX:UTF-8"

\documentclass{resume}
\usepackage{zh_CN-Adobefonts_external} % Simplified Chinese Support using external fonts (./fonts/zh_CN-Adobe/)
% \usepackage{NotoSansSC_external}
% \usepackage{NotoSerifCJKsc_external}
% \usepackage{zh_CN-Adobefonts_internal} % Simplified Chinese Support using system fonts
\usepackage{linespacing_fix} % disable extra space before next section
\usepackage{cite}

\begin{document}
\pagenumbering{gobble} % suppress displaying page number

\name{黄锋}

\basicInfo{
  \email{saintshaga@gmail.com} \textperiodcentered\ 
  \phone{(+86) 159-0040-8354}
  \textperiodcentered\ 
  \faMapMarker\ {上海}
%  \linkedin[billryan8]{https://www.linkedin.com/in/billryan8}
}
 
\section{\faGraduationCap\  教育背景}
\datedsubsection{\textbf{南京大学}}{2005 -- 2012}
{\textit{本科}(Top 10\%) \& \textit{硕士研究生}} - {电子信息科学与工程}

\section{\faCogs\ 技术经验简介}
% increase linespacing [parsep=0.5ex]
\begin{itemize}[parsep=0.5ex]
  \item 主要从事基于Java, Spring Boot, Tomcat的分布式开发,有多次Java服务器性能优化的经验
  \item 对于微服务的开发、测试、部署和监控有一定见解
  %各个方面都有哪些心得呢?这边得准备下
  \item 熟悉商用软件(toB)业务的开发和客户技术支持,有带领团队经验,擅长根据业务构建类结构
  %商用软件的开发跟对私业务有什么区别?在商用软件业务的客户支持有什么特别?
\end{itemize}


\section{\faUsers\ 工作经历}
\datedsubsection{\textbf{上海万革始应用软件(Works Applications)} Collaboration-Timeline}{2018 年7月 -- 至今}
\role{项目Leader}{Tomcat, Spring Boot, AWS DynamoDB, AWS Lambda}
\begin{onehalfspacing}
带领小组解决新框架下的难点,并且开发了公共的库:
\begin{itemize}
  \item 实现高性能多语言库、基于Spring Security的Auth认证、Tenant库、把旧项目中的后端渲染引擎移植到Spring Boot中并改进其效率,避免在把项目从旧系统移植到新架构下前端大量的更改
  \item 制定并且实现项目使用AWS CloudFront+S3(CDN)、AWS Beanstalk(Tomcat)中的自动部署方案,环境从build到部署在半小时内实现无宕机升级
  \item 对项目主页做并发测试,定位出性能有问题的地方并且进行了优化,在改进了rest调用、后端渲染引擎效率等问题后,最终定位出CPU对并发的影响(没有改进业务逻辑的情况下)。优化后,单个request的响应时间从约1.5s优化到约400ms,单instance的性能提高到150并发约2s内响应
\end{itemize}
%引入tracer机制,使得性能最终更加容易
\end{onehalfspacing}

\datedsubsection{\textbf{上海万革始应用软件(Works Applications)} HUE Announcement}{2016年5月 -- 2018年6月}
\role{项目Manager}{Spring, Cassandra, Kafka}
\begin{onehalfspacing}
带领小组使用spring, cassandra, kafka开发了Announcement, Notice, Advertisement等公司内部公告的功能
\begin{itemize}
  \item 根据业务需求做Cassandra的建模,保证数据库高性能访问,对于Cassandra的使用和原理有深入了解
  \item 根据业务构建系统类图,使得代码的层次结构更加清晰。解决开发中的难题,比如在业务复杂的情况下把数据分发给大量的用户。
  \item 带领小组成功完成项目并上线,保持组员关系融洽,在任职期间无小组成员离职
\end{itemize}
\end{onehalfspacing}

\datedsubsection{\textbf{上海万革始应用软件(Works Applications)} CIM Connector}{2012年8月 -- 2016年4月}
\role{软件工程师}{Tomcat, Spring, Hibernate, Axis2}
\begin{itemize}
  \item 使用Axis2开发GoogleApps Connector, Office365 Connector, IntraMart Connector, 并且维护和升级已经存在的Connector. 其中Office365 Connector在没有Java API的情况下使用JNI, C++和C\# 调用PowerShell Command完成
  \item batch性能优化:优化batch的执行方式,并且使用JDBC的batch执行,以及GoogleApps API的特点,把batch的执行速度提高了4倍
  \item 使用Seasar+S2Hibernate(Spring+Hibernate)开发Post Check功能来确保目标系统中数据的正确性
  \item 从无到有搭建了用于开发和测试用的目标系统(\textit{SAP})
\end{itemize}

% Reference Test
%\datedsubsection{\textbf{Paper Title\cite{zaharia2012resilient}}}{May. 2015}
%An xxx optimized for xxx\cite{verma2015large}
%\begin{itemize}
%  \item main contribution
%\end{itemize}


%\section{\faHeartO\ 获奖情况}
%\datedline{\textit{第一名}, xxx 比赛}{2013 年6 月}
%\datedline{其他奖项}{2015}

%\section{\faInfo\ 其他}
% increase linespacing [parsep=0.5ex]
%\begin{itemize}[parsep=0.5ex]
 % \item 技术博客: http://blog.yours.me
  %\item GitHub: https://github.com/username
  %\item 语言: 英语 - 熟练(TOEFL xxx)
%\end{itemize}

%% Reference
%\newpage
%\bibliographystyle{IEEETran}
%\bibliography{mycite}
\end{document}
