% !TEX program = xelatex

\documentclass{resume}
%\usepackage{zh_CN-Adobefonts_external} % Simplified Chinese Support using external fonts (./fonts/zh_CN-Adobe/)
%\usepackage{zh_CN-Adobefonts_internal} % Simplified Chinese Support using system fonts

\begin{document}
\pagenumbering{gobble} % suppress displaying page number

\name{Feng Huang}

\basicInfo{
  \email{saintshaga@gmail.com} \textperiodcentered\ 
  \phone{(+86) 159-0040-8354} \textperiodcentered\ 
  \faMapMarker \ {Shanghai}}

\section{\faGraduationCap\ Education}
\datedsubsection{\textbf{Nanjing University (NJU)}, Nanjing, China}{2005 -- 2012}
\textit{Bachelor}(Top 10\%) \& \textit{Master} in Electronics Engineering (EE)

\section{\faCogs\ Self Technical Introduction}
\begin{itemize}[parsep=0.5ex]
  \item Mainly engage in development of Java, Spring Boot, Tomcat, Distributed Database. Experienced in performance improving of Java server
  \item Got some ideas in development, test, deployment and monitor on MSA
  \item Be familiar with development and support of business software. Experienced as group leader
\end{itemize}

\section{\faUsers\ Experience}
%\datedsubsection{\textbf{Works Applications} Collaboration-Common \& Timeline}{2018.10 -- Present}
\datedsubsection{\textbf{Works Applications} Collaboration-Timeline}{2018.07 -- Present}
\role{Group Leader}{Tomcat, Spring Boot, AWS DynamoDB, AWS Lambda, AWS SQS}
Lead group to solve hard problems in new architecture, and develop some common libraries:
\begin{itemize}
  \item Implements high performance multi-language library, authentication library based on Spring Security, Multi-Tenant library. Transplant the backend render engine to Spring Boot and improve its efficiency, which avoids lots of modifications in front side when transplanting the project to new architecture
  \item Make and implement the deployment plan based on AWS CloudFront + S3(CDN) and AWS Beanstalk. The upgrading of environment takes less than half an hour from build to deploy successfully, without downtime
  \item Make load test to system to find out the bottleneck. After improving some bottleneck such as Rest Calling, backend rendering, not enough CPU is positioned. After performance improvement(in the condition that business logic is not improved), cost of one request is reduced from about 1.5s to 400ms, while performance of one instance is improved to about 2s response with 150 concurrency.
\end{itemize}

%\datedsubsection{\textbf{Works Applications} Collaboration-Performance Research}{2018.07 -- 2018.09}
%\role{Group Leader}{Tomcat, Spring, Kubernates, Cassandra}
%Lead group to research the performance bottleneck of our product(called HUE). After load test, we positioned problems of filter, multi-language library, Rest client, and so on.

\datedsubsection{\textbf{Works Applications} HUE Announcement}{2016.05 -- 2018.06}
\role{Group Manager}{Spring, Cassandra, Kafka}
Responsible for the product of distributing important information inside of company, for which Announcement, Notice and Advertisement are designed and developed.
\begin{itemize}
  \item Design Cassandra model according to business, which ensure efficient access to database. Know deeply how Cassandra runs
  \item Solve critical problems during development, such as distributing messages to large mount of targets in the case that business logic is complicated
  \item Finish developing all features and release the product in time. During my management period, team keeps good relationship, and no group member resigned
\end{itemize}

\datedsubsection{\textbf{Works Applications} CIM Connector}{2012.08 -- 2016.04}
\role{Software Engineer}{Spring, Hibernate, Axis2}

\begin{itemize}
  \item Develop multiple connectors(GoogleApps, Office 365, and so on) using Axis2, and maintain existed connectors. When developing Office365 connector, there was no Java API at that time. I use JNI, C++ and C\# to call PowerShell Command to implement the connector
  \item Improve the performance of batch execution. Change the way of executing batch, then make full use of JDBC's batch API and  GoogleApps API to accelerate the speed by 4 times
  \item Develop Post Check feature to make sure the accuracy of target system
  \item Establish target system(SAP) from 0 to 1, which helps a lot when maintaining and supporting SAP Connector
\end{itemize}

% Reference Test
%\datedsubsection{\textbf{Paper Title\cite{zaharia2012resilient}}}{May. 2015}
%An xxx optimized for xxx\cite{verma2015large}
%\begin{itemize}
%  \item main contribution
%\end{itemize}


%% Reference
%\newpage
%\bibliographystyle{IEEETran}
%\bibliography{mycite}
\end{document}
